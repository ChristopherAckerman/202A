% Created 2020-10-30 Fri 14:45
% Intended LaTeX compiler: pdflatex
\documentclass[11pt]{article}
\usepackage[utf8]{inputenc}
\usepackage[T1]{fontenc}
\usepackage{graphicx}
\usepackage{grffile}
\usepackage{longtable}
\usepackage{wrapfig}
\usepackage{rotating}
\usepackage[normalem]{ulem}
\usepackage{amsmath}
\usepackage{textcomp}
\usepackage{amssymb}
\usepackage{capt-of}
\usepackage{hyperref}
\usepackage{amsthm}
\usepackage{url}
\usepackage[margin=1.25in]{geometry}
\usepackage{hyperref}
\usepackage[dvipsnames]{xcolor}
\usepackage{booktabs}
\usepackage{enumitem}
\newtheorem*{definition}{Definition}
\newtheorem*{example}{Example}
\newtheorem*{theorem}{Theorem}
\newtheorem*{corollary}{Corollary}
\newtheorem*{exercise}{Exercise}
\newtheorem*{problem}{Problem}
\newtheorem{question}{Question}
\newcommand{\gr}{\textcolor{ForestGreen}}
\newcommand{\rd}{\textcolor{red}}
\newcommand{\R}{\mathbb{R}}
\newcommand{\p}{\mathbb{P}}
\newcommand{\frall}{\ \forall}
\author{Chris Ackerman}
\date{\today}
\title{Econ202A Homework \#1}
\hypersetup{
 pdfauthor={Chris Ackerman},
 pdftitle={Econ202A Homework \#1},
 pdfkeywords={},
 pdfsubject={},
 pdfcreator={Emacs 28.0.50 (Org mode 9.3)}, 
 pdflang={English}}
\begin{document}

\maketitle
\tableofcontents

\newpage

\section{Question 1}
\label{sec:orgf91538a}

\begin{enumerate}[label=(\alph*)]
\item
\begin{align*}
\max &\sum^\infty_{t = 0} 0.99 \log c_t \\
\text{s.t. } c_t + k_{t + 1} &\le k_t^{0.36}\\
%\implies c_t &= k^{0.36}_t - k_{t + 1}\\
k_0 & \text{ given}\\
\intertext{We can formulate this as a dynamic programming problem,}
V(k) &= \max_{k'} [\log(c_t) + 0.99 V(k')]\\
\intertext{Starting from $v_0(k) = 0$,}
\max \log c_t \text{ s.t. } c_t &\le k^{0.36}_t - k_{t + 1}\\
\intertext{We know, from the functional forms for production and utility, that the solution takes the form}
V(k) &= A + \frac{0.36}{1 - 0.36 \cdot 0.99} \log (k),\\
\implies \gr{k'} &= \gr{0.36 \cdot 0.99 k^{0.36}}
\end{align*}
\item
\begin{align*}
% \overline{k} &= 0.99 \cdot 0.36 k^{0.36}\\
% \overline{k}^{0.64} &= 0.3564\\
% \gr{\overline{k}} &= \gr{0.19948}
\intertext{Starting from the Bellman Equation}
V(k) &= \max_{k'} [\log(c_t) + 0.99 V(k')]\\
\intertext{We can take the first order condition with respect to $k'$ and then apply the envelope theorem to get}
\rd{\frac{1}{c_t}} &= \beta \rd{\frac{1}{c_t}} 0.36 k^{0.36 - 1}\\
\gr{\overline{k}} &=  \gr{0.19948}
\end{align*}
\newpage
\item Matlab code:
\begin{verbatim}
beta = 0.99;
delta = 0;
epsilon = 1e-6;
alpha = 0.36;
delta = 1;

ks = ((1 - beta*(1 - delta))/(alpha * beta))^(1/(alpha  - 1));

kmin = 0;
kmax = 0.2;
grid_size = 10000;

dk = (kmax-kmin)/(grid_size - 1);
kgrid = linspace(kmin, kmax, grid_size);
v = zeros(grid_size, 1);
dr = zeros(grid_size, 1);
norm = 1;

while norm > epsilon;
    for i=1:grid_size
        tmp = (kgrid(i)^alpha + (1 - delta)*kgrid(i) - kmin);
        imax = min(floor(tmp/dk) + 1, grid_size);
        
        c = kgrid(i)^alpha + (1 - delta)*kgrid(i) - kgrid(1:imax);
        util = log(c);
        [tv(i), dr(i)] = max(util + beta*(1:imax));
    end;
    norm = max(abs(tv - v));
    v = tv;
end;
\end{verbatim}
This returns a steady-state capital stock value of \gr{0.1995}

\newpage
\item[(e)] Since there is depreciation, I increased the maximum capital stock and used enough more grid points than recommended in the problem.

Matlab code 
\begin{verbatim}
beta = 0.99;
epsilon = 1e-6;
alpha = 0.36;
delta = .02;

ks = ((1 - beta*(1 - delta))/(alpha * beta))^(1/(alpha  - 1));

kmin = 0;
kmax = 50;
grid_size = 10000;

dk = (kmax-kmin)/(grid_size - 1);
kgrid = linspace(kmin, kmax, grid_size);
v = zeros(grid_size, 1);
dr = zeros(grid_size, 1);
norm = 1;

while norm > epsilon;
    for i=1:grid_size
        tmp = (kgrid(i)^alpha + (1 - delta)*kgrid(i) - kmin);
        imax = min(floor(tmp/dk) + 1, grid_size);
        
        c = kgrid(i)^alpha + (1 - delta)*kgrid(i) - kgrid(1:imax);
        util = log(c);
        [tv(i), dr(i)] = max(util + beta*(1:imax));
    end;
    norm = max(abs(tv - v));
    v = tv;
end;\end{verbatim}
This returns a steady-state capital stock value of \gr{48.2992}
\end{enumerate}

\newpage
\section{Question 2}
\label{sec:orgf6fb3b4}
\begin{enumerate}[label=(\alph*)]
  \item
\begin{align*}
V(x, k) &= \max_{k'}[u(e^{z_t} k_t^\theta - k_{t + 1}) + \beta \int_{z'} V(z', k') dG(z')]\\
\text{s.t. } c_t + k_{t + 1} &\le e^{z_t} k_t^\theta\\
\intertext{From the common functional forms that we know, this implies the law of motion}
k' &= E + F\log + Gz
\end{align*}
\item Since we have two states, we need to set up two functions
\begin{align*}
v_l(k) &\equiv v(z_l, k)\\
v_h(k) &\equiv v(z_h, k)\\
Tv_l(k) &= \max_{k'}(u(z_l)f(k) - k') + \beta [qv_l(k') + (1 - q)v_h(k')]\\
Tv_h(k) &= \max_{k'}(u(z_h)f(k) - k') + \beta [(1 - q)v_l(k') + q v_h(k')]\\
\intertext{For this two-state Markov problem, the functional form is}
v^l_0(k) &= E_l F \log(k)\\
v^h_0(k) &= E_h F \log(k)\\
\intertext{And we split our guesses across states.}
k' &= \left\{
\begin{array}{ll}
E_l F \log(k) &\text{ if } z = z_l\\
E_h F \log(k) &\text{ if } z = z_h\\
\end{array}
\right.
\end{align*}
  \end{enumerate}

\newpage
\section{Question 3}
\label{sec:org0b176f4}
\begin{enumerate}[label=(\alph*)]
\item 
\begin{align*}
V(k) &= \max_{c, i, h_1, h_2, k_1, k'} \{u(c, 1 - h_1 - h_2) + \beta V(k')\}\\
\text{s.t. } i &= f_1(k_1, h_1)\\
c &= f_2(k - k_1, h_2)\\
k' &= i + (1 - \delta)k
\end{align*}
\item Our RCE in this case has four components and four constraints:
\begin{enumerate}[label=\arabic*.]
\item Household decision rules $c(K, k)$, $h(K, k)$, and $k'(K, k)$ along with a value function $V(K, k)$
\item Two decision rules for each firm, $k^c(K)$, $h^c(K)$, $k^i(K)$, $h^i(K)$
\item Three price functions, $p_c(K)$, $r(K)$, $w(K)$
\item A law of motion for the aggregate capital stock $K' = \hat{G}(K)$
\end{enumerate}
such that
\begin{enumerate}[label=(\alph*)]
\item Given the price functions and perceived aggregate law of motion, the household decision rules solve the household's problem.
\item Given the price functions, $k(K)$ and $h(K)$ solve the firm's problem (for each firm)
\item Markets clear:
\begin{align*}
h(K, K) &= h^c(K) + h^i(K)\\
K &= k^c(K) + k^i(K)\\
k'(K, K) - (1 - \delta)K &= F_1(k^i, h^i)
\end{align*}
\item $\hat{G}(K) = k'(K, K)$
\end{enumerate}
\end{enumerate}
\end{document}